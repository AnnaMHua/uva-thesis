  \chapter{The High Resolution Spectrometer Optimization $\simeq 30$}

PRex-II Experiment was operated on Jefferson Lab High-Resolution Spectrometer(JLab). The kinematics of the scattered electrons have to be measured on the Vertical Drift Chambers which are installed after the magnetic chains. All the measurements have to be converted back to the target coordination system in order to get the result. This chapter will discuss the detail of JLab HRS calibration, the calculation of the 4-momentum transfer square, and the GEM detector analysis. 


\section{High-Resolution Spectrometer Optics Calibration}

(a more detailed apparatus could/will be found in another chapter describing the detailed experiment configuration)

The Vertical Drift Chamber provides the HRS spatial resolution capability. Each of the two HRS spectrometers contains two Vertical Drift Chamber(VDC). each chamber has 750 wires in two (U, V) planes, position resolution of 100 um per plane. Four VDC chamber layers of wires can provide partial spatial location $x$, $y$ together with particle angle $\theta$ and $\phi$. The kinematics, like scattered angle and scattered momentum, need to derive from the four direct measurable parameters(VDC $x$, $y$,$\theta$,$\phi$). The calibration procedure can be categorized in three steps:

\begin{itemize}
    \item VDC drift time $t_0$ calibration
    \item VDC geometry constant calibration 
    \item spatial and angular calibration
    \item momentum calibration
\end{itemize}

Vertical Drift Chamber measures the spatial position by measuring the particle drift time. $t_0$ serves as the starting time of the VDC counter. Carefully calibrated $t_0$ is critical to the spatial and angular resolution of the VDC. To simplify the regression model and easily converge, the spatial, angular, and momentum parameters are calibrated from the focal plane. The VDC geometry constants are used to convert the direct measurements on the VDC plane to the focal plane. Before going into the detail of the calibration, here first introduce the coordination system used in the following contest. 

\subsection{coordinate system}

[Introduce the different coordinate systems and also the derivative of different coordination systems. The conversion from the VDC to the rotated focal plane is also the equation used for the VDC geo-constant calibration ]


To simplify the calculation, couples of different coordinate systems are used. In this section, five coordinate systems and their directive relations are briefly introduced. A more detailed coordinate system of Jefferson Lab Hall A can be found in the ESPACE User Manual\cite{espace2002manual}.  

\subsubsection{Hall Coordinate System}
The center of the Hall Coordinate system is defined as the intersection of the electron beam and the vertical symmetry axis of the target system. $z$ is along the beam dump direction. $y$-axis is pointing vertically up against gravity.

\todo{add HCS plot}

\subsubsection{Target Coordinate System}

Each High-Resolution Spectrometer(HRS) has its own target coordinate system. Ideally, the center of the target coordinate system coincides with the center of the Hall A coordinate system. The distance from the hall A center to the midpoint of the center of the sieve slit hole is defined as $Z_0$ ($Z^{HRSE}_0 = 1.181 m$, $Z^{HRSH}_0 = 1.178 m$). The center of the target coordinate system is defined as $Z_0$ distance from the sieve slit(ref to chapter xx section xx) with $z$-axis perpendicular to the center of the sieve slit plane pointing to the beam direction. The $x$-axis is parallel to the sieve slit surface with $x$ pointing vertically down. The $\theta$ and $\phi$ angle is defined as $\frac{dx_{tg}}{Z_0}$ and $\frac{dy_{tg}}{Z_0}$ respectively.

\todo{add TCS plot}


\subsubsection{Detector Coordinate System}

Each of the VDC chambers has 368 wires placed at $\pm 45 ^{\circ} u/v$ direction. The center of the detector coordinate system is defined as the intersection of wire 184 of the VDC1 U1 plane and the perpendicular projection of wire 184 in the VDC1 V1 plane onto the VDC1 U1 plane. $x$-axis is along the long edge of the VDC plane, while y is along the short edge of the VDC plane and $z$ is perpendicular to the VDC plane and pointing upwards. 

\todo{add vdc, DCS plot}

\subsubsection{Transport Rotation Coordinate System at the focal plane}
The transport coordinate system at the focal plane is generated by rotating the detector coordinate system by $45^{\circ}$ along the y-axis. Ideally the z-axis along the central beam ray. 

[derive of the TRCS]
\todo{add the plot of the transport coordinate system plot}


\subsubsection{Focal plane Coordinate System}




\subsection{the calculation of the coordinate system}

% Introduction:
% \begin{itemize}
%     \item apparatus
%     \item why we need to Optics 
%     \item how to do optics (linear regression))
%     \item a brief of the result
% \end{itemize}


% \lipsum[1-20]
\subsection{Apparatus[deprecated]}

Move to a separate chapter describing the apparatus and experiment details

\begin{itemize}
    \item Overview of the HRS structure
    \item septum magnet 
    \item Vertical drift chamber 
    \item GEM detector
    \item supporting equipment used for optimization only sieve slide

    \item coordination system of the Jefferson Lab Hall A
    \item coordination system of the HRS
\end{itemize}


\subsection{HRS model}

\begin{itemize}
    \item mathematical model of the HRS
    \item constant parameter optimization
    \item Vertical Drift Chamber time optimization
    \item carbon calibration(dashed lines in the VDC spectrometer)
    \item math why higher order contribute, a discussion notes
    \item feature selection techniques
    \item linear regression
    \item result validation
    \item result and discussion
\end{itemize}

\subsection{VDC geo-constant calibration}
\subsection{Raster calibration}

% \subsubsection{Linear Regression and mathematics approximation}




% \subsection{}section{feature selection}

% \subsubsection{general consideration in regression}

% \subsubsection{features in regression}

% \subsubsection{LASSO feature reduction regression}

% \subsubsection{ridge feature reduction regression}

% \subsection{autoML and more}

% \section{Momentum optimization}

% \section{Regression Result validation}

% \subsection{carbon result check}

% \subsubsection{first, second, third momentum check}

% \section{momentum reconstruction}

% \section{scattered angle reconstruction and correction}

% \section{a second thought/attempt on the regression}

% \subsection{another more complicated model}