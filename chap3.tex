  \chapter{The High Resolution Spectrometer Optimization(30)}

Introduction:
\begin{itemize}
    \item apparatus
    \item why we need to Optics 
    \item how to do optics (linear regression))
    \item a brief of the result
\end{itemize}


% \lipsum[1-20]
\section{Apparatus}

\begin{itemize}
    \item Overview of the HRS structure
    \item septum magnet 
    \item Vertical drift chamber 
    \item GEM detector
    \item supporting equipment used for optimization only sieve slide

    \item coordination system of the Jefferson Lab Hall A
    \item coordination system of the HRS
\end{itemize}


\section{HRS model}

\begin{itemize}
    \item mathematic model of the HRS
    \item constant parameter optimization
    \item VDC drift time optimization
    \item carbon calibration(dashed lines in the VDC spectrometer)
    \item math why higher order contribute, a discussion notes
    \item feature selection technics
    \item linear regression
    \item result validation
    \item result and discussion
\end{itemize}

\subsection{Linear Regression and mathematics approximation}




\section{feature selection}

\subsection{general consideration in regression}

\subsection{features in regression}

\subsection{LASSO feature reduction regression}

\subsection{ridge feature reduction regression}

\subsection{autoML and more}

\section{Momentum optimization}

\section{Regression Result validation}

\subsection{carbon result check}

\subsubsection{first, second, third momentum check}

\section{momentum reconstruction}

\section{scattered angle reconstruction and correction}

\section{a second thought/attempt on the regression}

\subsection{another more complicated model}