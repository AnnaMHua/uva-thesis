\chapter{The Theory of PRex Experiment}
% \todo[inline]{Separate the theory and the experiment part into two chapters}

The RMS radius of the  neutron distribution in a heavy nucleus  $R_N$ provides an important test of nuclear theory. Furthermore   $R_N$ is used in the determination of  the density dependence of symmetry energy of neutron rich matter; this dependence is an  important input in   neutron star structure, heavy iron collision and atomic parity violation experiment calculations. In the past hadron scattering experiments with with pion, proton or anti-proton beams have been used to determine the neutron radii of heavy nuclei. However, these measurements suffer from uncertainties associated with the probe particle and the target nucleus. Electron scattering provides a model independent probe of nuclear radii.  However, in electron scattering, the measurement of neutron distribution in a nucleus  is much harder than the measurement of the proton distribution  since the neutron is uncharged. Because the  neutron weak charge is much large than that of the proton, PRex-II  used the parity violating weak neutral interaction to probe the neutron distribution in the  ${^{208}}Pb$ nucleus, thus measuring the RMS neutron radius with high  accuracy. The PRex-II experiment was performed from June to September 2019 in Jefferson lab experimental hall A using the High Resolution Spectrometer (HRS) pair. 

\section{electron scattering}

In 1961, Robert Hofstadter wins Nobel Price for his pioneering studies of structure of nucleons with  electron scattering. Accelerated electron beams are widely used for study the inner structure of the nucleons (and neutrons) since then. Those experiment can provide different level structure information of nucleons given different incident electron energy. At the range where the energy of electrons are very low, $\lambda \gg r_p$, here $r_p$ is the radius of the proton, $\lambda$ is the electron wavelength, the scattering is equivalent to scattered over point like spin-less object. At low electron energy range where $\lambda ~r_p$, the scattering is equivalent to scattering over charged object. When the electron wavelength $\lambda < r_p$, the electron scattering will be able to see the substructures. At very high energy range where $\lambda \ll r_p$, the electrons is equivalent to scattering over the sea of quarks and gluons of the protons. 

In one photon exchange approximation, the transition current of electron can be write as:
$$
j^\mu = - e\bar{\mu}_e(k')\gamma^\mu\mu_e(k)e^{i\vec{q}\vec{x}}
$$

The  transition current for nucleon can be write as:

$$
J^\mu = e\bar{\mu}\Gamma^\mu(p)e^{i\vec{q}\vec{x}}$$
$$

\todo[inline]{Add the Feynman diagram}

In one photon exchange approximation, the cross section of electron scattered over spin-0 particle would be:

$$
\frac{d\sigma}{d\Omega} = \frac{d\sigma}{d\Omega}|_{mott}*|F(Q^2)|^2
$$

\todo[inline]{current result of proton radius (PRad)}

\subsection{e-p scattering}
\subsection{e-n scattering}
\subsection{form factor}

\section{Weak interaction and weak current}
\section{Parity Violation}
\section{Parity Violation Asymmetry}
\todo[inline]{derive the asymmetry in term of form factor}
\section{Rich Physics Behind the PRex Experiment}
\subsection{EOS of Neutron rich matter}