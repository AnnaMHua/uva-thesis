\chapter{The Theory and Experiment Setups}
\todo[inline]{Separate the theory and the experiment part into two chapters}

The RMS radius of the  neutron distribution in a heavy nucleus  $R_N$ provides an important test of nuclear theory. Furthermore   $R_N$ is used in the determination of  the density dependence of symmetry energy of neutron rich matter; this dependence is an  important input in   neutron star structure, heavy iron collision and atomic parity violation experiment calculations. In the past hadron scattering experiments with with pion, proton or anti-proton beams have been used to determine the neutron radii of heavy nuclei. However, these measurements suffer from uncertainties associated with the probe particle and the target nucleus. Electron scattering provides a model independent probe of nuclear radii.  However, in electron scattering, the measurement of neutron distribution in a nucleus  is much harder than the measurement of the proton distribution  since the neutron is uncharged. Because the  neutron weak charge is much large than that of the proton, PRex-II  used the parity violating weak neutral interaction to probe the neutron distribution in the  ${^{208}}Pb$ nucleus, thus measuring the RMS neutron radius with high  accuracy. The PRex-II experiment was performed from June to September 2019 in Jefferson lab experimental hall A using the High Resolution Spectrometer (HRS) pair. 

\section{elastic electron scattering}



\subsection{e-p scattering}
\subsection{e-n scattering}
\section{Weak current}
\section{Parity Violation}
\section{Parity Violation Asymmetry}
\section{Rich Physics Behind the PRex Experiment}
\subsection{EOS of Neutron rich matter}